\documentclass[]{article}
\usepackage{lmodern}
\usepackage{amssymb,amsmath}
\usepackage{ifxetex,ifluatex}
\usepackage{fixltx2e} % provides \textsubscript
\ifnum 0\ifxetex 1\fi\ifluatex 1\fi=0 % if pdftex
  \usepackage[T1]{fontenc}
  \usepackage[utf8]{inputenc}
\else % if luatex or xelatex
  \ifxetex
    \usepackage{mathspec}
  \else
    \usepackage{fontspec}
  \fi
  \defaultfontfeatures{Ligatures=TeX,Scale=MatchLowercase}
\fi
% use upquote if available, for straight quotes in verbatim environments
\IfFileExists{upquote.sty}{\usepackage{upquote}}{}
% use microtype if available
\IfFileExists{microtype.sty}{%
\usepackage[]{microtype}
\UseMicrotypeSet[protrusion]{basicmath} % disable protrusion for tt fonts
}{}
\PassOptionsToPackage{hyphens}{url} % url is loaded by hyperref
\usepackage[unicode=true]{hyperref}
\hypersetup{
            pdftitle={transitiveInference},
            pdfauthor={Petra Borovska},
            pdfborder={0 0 0},
            breaklinks=true}
\urlstyle{same}  % don't use monospace font for urls
\usepackage[margin=1in]{geometry}
\usepackage{color}
\usepackage{fancyvrb}
\newcommand{\VerbBar}{|}
\newcommand{\VERB}{\Verb[commandchars=\\\{\}]}
\DefineVerbatimEnvironment{Highlighting}{Verbatim}{commandchars=\\\{\}}
% Add ',fontsize=\small' for more characters per line
\usepackage{framed}
\definecolor{shadecolor}{RGB}{248,248,248}
\newenvironment{Shaded}{\begin{snugshade}}{\end{snugshade}}
\newcommand{\KeywordTok}[1]{\textcolor[rgb]{0.13,0.29,0.53}{\textbf{#1}}}
\newcommand{\DataTypeTok}[1]{\textcolor[rgb]{0.13,0.29,0.53}{#1}}
\newcommand{\DecValTok}[1]{\textcolor[rgb]{0.00,0.00,0.81}{#1}}
\newcommand{\BaseNTok}[1]{\textcolor[rgb]{0.00,0.00,0.81}{#1}}
\newcommand{\FloatTok}[1]{\textcolor[rgb]{0.00,0.00,0.81}{#1}}
\newcommand{\ConstantTok}[1]{\textcolor[rgb]{0.00,0.00,0.00}{#1}}
\newcommand{\CharTok}[1]{\textcolor[rgb]{0.31,0.60,0.02}{#1}}
\newcommand{\SpecialCharTok}[1]{\textcolor[rgb]{0.00,0.00,0.00}{#1}}
\newcommand{\StringTok}[1]{\textcolor[rgb]{0.31,0.60,0.02}{#1}}
\newcommand{\VerbatimStringTok}[1]{\textcolor[rgb]{0.31,0.60,0.02}{#1}}
\newcommand{\SpecialStringTok}[1]{\textcolor[rgb]{0.31,0.60,0.02}{#1}}
\newcommand{\ImportTok}[1]{#1}
\newcommand{\CommentTok}[1]{\textcolor[rgb]{0.56,0.35,0.01}{\textit{#1}}}
\newcommand{\DocumentationTok}[1]{\textcolor[rgb]{0.56,0.35,0.01}{\textbf{\textit{#1}}}}
\newcommand{\AnnotationTok}[1]{\textcolor[rgb]{0.56,0.35,0.01}{\textbf{\textit{#1}}}}
\newcommand{\CommentVarTok}[1]{\textcolor[rgb]{0.56,0.35,0.01}{\textbf{\textit{#1}}}}
\newcommand{\OtherTok}[1]{\textcolor[rgb]{0.56,0.35,0.01}{#1}}
\newcommand{\FunctionTok}[1]{\textcolor[rgb]{0.00,0.00,0.00}{#1}}
\newcommand{\VariableTok}[1]{\textcolor[rgb]{0.00,0.00,0.00}{#1}}
\newcommand{\ControlFlowTok}[1]{\textcolor[rgb]{0.13,0.29,0.53}{\textbf{#1}}}
\newcommand{\OperatorTok}[1]{\textcolor[rgb]{0.81,0.36,0.00}{\textbf{#1}}}
\newcommand{\BuiltInTok}[1]{#1}
\newcommand{\ExtensionTok}[1]{#1}
\newcommand{\PreprocessorTok}[1]{\textcolor[rgb]{0.56,0.35,0.01}{\textit{#1}}}
\newcommand{\AttributeTok}[1]{\textcolor[rgb]{0.77,0.63,0.00}{#1}}
\newcommand{\RegionMarkerTok}[1]{#1}
\newcommand{\InformationTok}[1]{\textcolor[rgb]{0.56,0.35,0.01}{\textbf{\textit{#1}}}}
\newcommand{\WarningTok}[1]{\textcolor[rgb]{0.56,0.35,0.01}{\textbf{\textit{#1}}}}
\newcommand{\AlertTok}[1]{\textcolor[rgb]{0.94,0.16,0.16}{#1}}
\newcommand{\ErrorTok}[1]{\textcolor[rgb]{0.64,0.00,0.00}{\textbf{#1}}}
\newcommand{\NormalTok}[1]{#1}
\usepackage{graphicx,grffile}
\makeatletter
\def\maxwidth{\ifdim\Gin@nat@width>\linewidth\linewidth\else\Gin@nat@width\fi}
\def\maxheight{\ifdim\Gin@nat@height>\textheight\textheight\else\Gin@nat@height\fi}
\makeatother
% Scale images if necessary, so that they will not overflow the page
% margins by default, and it is still possible to overwrite the defaults
% using explicit options in \includegraphics[width, height, ...]{}
\setkeys{Gin}{width=\maxwidth,height=\maxheight,keepaspectratio}
\IfFileExists{parskip.sty}{%
\usepackage{parskip}
}{% else
\setlength{\parindent}{0pt}
\setlength{\parskip}{6pt plus 2pt minus 1pt}
}
\setlength{\emergencystretch}{3em}  % prevent overfull lines
\providecommand{\tightlist}{%
  \setlength{\itemsep}{0pt}\setlength{\parskip}{0pt}}
\setcounter{secnumdepth}{0}
% Redefines (sub)paragraphs to behave more like sections
\ifx\paragraph\undefined\else
\let\oldparagraph\paragraph
\renewcommand{\paragraph}[1]{\oldparagraph{#1}\mbox{}}
\fi
\ifx\subparagraph\undefined\else
\let\oldsubparagraph\subparagraph
\renewcommand{\subparagraph}[1]{\oldsubparagraph{#1}\mbox{}}
\fi

% set default figure placement to htbp
\makeatletter
\def\fps@figure{htbp}
\makeatother


\title{transitiveInference}
\author{Petra Borovska}
\date{21 June 2020}

\begin{document}
\maketitle

\section{General Set Up}\label{general-set-up}

clean environment and console

\begin{Shaded}
\begin{Highlighting}[]
\KeywordTok{cat}\NormalTok{(}\StringTok{"}\CharTok{\textbackslash{}014}\StringTok{"}\NormalTok{)}
\end{Highlighting}
\end{Shaded}



\begin{Shaded}
\begin{Highlighting}[]
\KeywordTok{rm}\NormalTok{(}\DataTypeTok{list =} \KeywordTok{ls}\NormalTok{())}
\end{Highlighting}
\end{Shaded}

Installing all necessary packages for data preparation

\section{Data Preparation}\label{data-preparation}

\begin{center}\rule{0.5\linewidth}{0.5pt}\end{center}

Data preparation, loding data set, basic adjustments

get current directory

\begin{Shaded}
\begin{Highlighting}[]
\KeywordTok{here}\NormalTok{()}
\end{Highlighting}
\end{Shaded}

\begin{verbatim}
## [1] "C:/Users/ibm/Documents/Results/R"
\end{verbatim}

list the files in the desired direcotry

\begin{Shaded}
\begin{Highlighting}[]
\KeywordTok{here}\NormalTok{(}\StringTok{"dataSets"}\NormalTok{)}
\end{Highlighting}
\end{Shaded}

\begin{verbatim}
## [1] "C:/Users/ibm/Documents/Results/R/dataSets"
\end{verbatim}

\begin{Shaded}
\begin{Highlighting}[]
\KeywordTok{list.files}\NormalTok{(}\KeywordTok{here}\NormalTok{(}\StringTok{"dataSets"}\NormalTok{))  ## just double check if it's correctly saved}
\end{Highlighting}
\end{Shaded}

\begin{verbatim}
## [1] "pilotA_sleep_tr.csv"  "pilotA_sleep_tr.xlsx" "pilotA_sleep_ts.csv" 
## [4] "pilotA_sleep_ts.xlsx" "pilotA_wake_tr.csv"   "pilotA_wake_tr.xlsx" 
## [7] "pilotA_wake_ts.csv"   "pilotA_wake_ts.xlsx"
\end{verbatim}

load the data for each data set

\begin{Shaded}
\begin{Highlighting}[]
\NormalTok{tr_sleep <-}\StringTok{ }
\StringTok{        }\KeywordTok{read.csv}\NormalTok{(}
                \KeywordTok{here}\NormalTok{(}
                        \StringTok{"dataSets"}\NormalTok{, }
                        \StringTok{"pilotA_sleep_tr.csv"}
\NormalTok{                )}
\NormalTok{        )}

\NormalTok{tr_wake <-}\StringTok{ }
\StringTok{        }\KeywordTok{read.csv}\NormalTok{(}
                \KeywordTok{here}\NormalTok{(}
                        \StringTok{"dataSets"}\NormalTok{, }
                        \StringTok{"pilotA_wake_tr.csv"}
\NormalTok{                )}
\NormalTok{        )}


\NormalTok{test_sleep <-}\StringTok{ }
\StringTok{        }\KeywordTok{read.csv}\NormalTok{(}
                \KeywordTok{here}\NormalTok{(}
                        \StringTok{"dataSets"}\NormalTok{, }
                        \StringTok{"pilotA_sleep_ts.csv"}
\NormalTok{                )}
\NormalTok{        )}

\NormalTok{test_wake <-}\StringTok{ }
\StringTok{        }\KeywordTok{read.csv}\NormalTok{(}
                \KeywordTok{here}\NormalTok{(}
                        \StringTok{"dataSets"}\NormalTok{, }
                        \StringTok{"pilotA_wake_ts.csv"}
\NormalTok{                )}
\NormalTok{        )}
\end{Highlighting}
\end{Shaded}

look at the data

\begin{Shaded}
\begin{Highlighting}[]
\NormalTok{## show the data}
\CommentTok{# tr_sleep}
\CommentTok{# tr_wake}
\CommentTok{# test_sleep}
\CommentTok{# test_wake}
\end{Highlighting}
\end{Shaded}

clean data, delete not useful rows from each data set Drop na only for
testing, where the length of tibble is the same, not do for training, it
varies and it will drop also useful rows

\begin{Shaded}
\begin{Highlighting}[]
\NormalTok{test_sleep =}\StringTok{ }\NormalTok{test_sleep }\OperatorTok\StringTok{ }\KeywordTok{drop_na}\NormalTok{()}
\NormalTok{test_wake =}\StringTok{ }\NormalTok{test_wake }\OperatorTok\StringTok{ }\KeywordTok{drop_na}\NormalTok{()}
\end{Highlighting}
\end{Shaded}

\begin{Shaded}
\begin{Highlighting}[]
\NormalTok{ts_sleep_R =}\StringTok{ }\NormalTok{test_sleep}
\NormalTok{ts_wake_R =}\StringTok{ }\NormalTok{test_wake}
\end{Highlighting}
\end{Shaded}

test\_wake with the most data, I guess use that need to get average from
each pair then I need to switch participants columns with rows

my columns are always:\\
pairType \_condition\_token letterPos1 \_condition\_token letterPos2
\_condition\_token key\_resp\_test.corr \_condition\_token \textless{}-
that might be different name with training key\_resp\_test.rt
\_condition\_token \textless{}- also might be different for training

to calculate frequencies - total scores: premise 50 1deg 20 2deg 10
anchor 10

\begin{Shaded}
\begin{Highlighting}[]
\NormalTok{premTot =}\StringTok{ }\DecValTok{50}
\NormalTok{oneDegTot =}\StringTok{ }\DecValTok{20}
\NormalTok{twoDegTot =}\StringTok{ }\DecValTok{10}
\NormalTok{anchorTot =}\StringTok{ }\DecValTok{10}
\end{Highlighting}
\end{Shaded}

\begin{Shaded}
\begin{Highlighting}[]
\NormalTok{colName_w_ts =}
\StringTok{        }\NormalTok{ts_wake_R }\OperatorTok
\StringTok{        }\KeywordTok{colnames}\NormalTok{(.)}
\NormalTok{colNum_w_ts =}
\StringTok{        }\NormalTok{ts_wake_R }\OperatorTok
\StringTok{        }\KeywordTok{ncol}\NormalTok{()}
\end{Highlighting}
\end{Shaded}

\begin{Shaded}
\begin{Highlighting}[]
\CommentTok{# ts_wake_6_select =}
\CommentTok{#         ts_wake_R %>%}
\CommentTok{#         select(}
\CommentTok{#                 c(1:5)}
\CommentTok{#         )}
\CommentTok{#         }
\CommentTok{# }
\CommentTok{# ts_wake_6_multiple =}
\CommentTok{#         ts_wake_6_select %>%}
\CommentTok{#         group_by(}
\CommentTok{#                 pairType_W_6}
\CommentTok{#         ) %>%}
\CommentTok{#         summarize(}
\CommentTok{#                 tibble(}
\CommentTok{#                 total_W_6 = sum(key_resp_test.corr_W_6), }
\CommentTok{#                 rt_mean_W_6 = mean(key_resp_test.rt_W_6)}
\CommentTok{#                 )}
\CommentTok{#         ) %>%}
\CommentTok{#         add_column(max_W_6 = c(10, 20, 50, 10))}
\end{Highlighting}
\end{Shaded}

\section{Functions}\label{functions}

write a funtion which gives me summary dfs, containing total number per
pair type, mean rt per pair type and total number of pair types per
group

parameters that needs to be inserted: type of data set which columns
from the data set, 5 columns for each participant - maybe create a range
variables for each participant

does not have to be insert as parameter, but need to be set up\\
vary pair type - it's always first column of new data set, so replace
the column name it does not have to vary proportion column, again
replace by number - it does not have to vary mean column - again replace
by number - it does not have to vary

think about that when training - na s must be drop during the function
call - when selection going on - so between step

For the function is good to make copy

\begin{Shaded}
\begin{Highlighting}[]
\NormalTok{summary_subj_ts_f <-}\StringTok{ }\ControlFlowTok{function}\NormalTok{(dataSet, s, e)\{}
        
\NormalTok{        s =}\StringTok{ }\KeywordTok{as.integer}\NormalTok{(s)}
\NormalTok{        e =}\StringTok{ }\KeywordTok{as.integer}\NormalTok{(e)}

\NormalTok{        dataSet_select =}\StringTok{ }\NormalTok{dataSet[, }\KeywordTok{c}\NormalTok{(s}\OperatorTok{:}\NormalTok{e)]}
\NormalTok{        dataSet_select =}\StringTok{ }\KeywordTok{as_tibble}\NormalTok{(dataSet_select)}
        
        
\NormalTok{        origColNames =}\StringTok{ }\KeywordTok{colnames}\NormalTok{(dataSet_select)[}\KeywordTok{c}\NormalTok{(}\DecValTok{1}\NormalTok{, }\DecValTok{4}\NormalTok{, }\DecValTok{5}\NormalTok{)]}
        
        \KeywordTok{colnames}\NormalTok{(dataSet_select) <-}\StringTok{ }\KeywordTok{c}\NormalTok{(}\StringTok{"pairType"}\NormalTok{, }\StringTok{"letterPos1"}\NormalTok{, }\StringTok{"letterPos2"}\NormalTok{,}
                                      \StringTok{"key_resp_test.corr"}\NormalTok{, }\StringTok{"key_resp_test.rt"}\NormalTok{)}
        
\NormalTok{        dataSet_multiple =}
\StringTok{                }\NormalTok{dataSet_select }\OperatorTok
\StringTok{                }\KeywordTok{group_by}\NormalTok{(}
\NormalTok{                        pairType}
\NormalTok{                ) }\OperatorTok
\StringTok{                }\KeywordTok{summarize}\NormalTok{(}
                        \KeywordTok{tibble}\NormalTok{(}
                        \DataTypeTok{total_corr =} \KeywordTok{sum}\NormalTok{(key_resp_test.corr), }
                        \DataTypeTok{mean_rt =} \KeywordTok{mean}\NormalTok{(key_resp_test.rt)}
\NormalTok{                        )}
\NormalTok{                )}

        \KeywordTok{colnames}\NormalTok{(dataSet_multiple) <-}\StringTok{ }\NormalTok{origColNames}
        
        
\NormalTok{        dataSet_multiple_t =}\StringTok{ }
\StringTok{                }\NormalTok{dataSet_multiple }\OperatorTok
\StringTok{                }\KeywordTok{gather}\NormalTok{(}\DataTypeTok{key =}\NormalTok{ corr_cond_subj, }\DataTypeTok{value =}\NormalTok{ value, }\DecValTok{2}\OperatorTok{:}\KeywordTok{ncol}\NormalTok{(dataSet_multiple)) }\OperatorTok\StringTok{ }
\StringTok{                }\KeywordTok{spread_}\NormalTok{(}\DataTypeTok{key =} \KeywordTok{names}\NormalTok{(dataSet_multiple)[}\DecValTok{1}\NormalTok{],}\DataTypeTok{value =} \StringTok{'value'}\NormalTok{)}
        
        
\NormalTok{        dataSet_multiple_t_head =}\StringTok{ }\KeywordTok{slice_head}\NormalTok{(dataSet_multiple_t)}
\NormalTok{        dataSet_multiple_t_tail =}\StringTok{ }\KeywordTok{slice_tail}\NormalTok{(dataSet_multiple_t)}
\NormalTok{        dataSet_multiple_t_less =}
\StringTok{                }\NormalTok{dataSet_multiple_t_tail }\OperatorTok
\StringTok{                }\KeywordTok{select}\NormalTok{(}\OperatorTok{-}\DecValTok{1}\NormalTok{) }\OperatorTok
\StringTok{                }\KeywordTok{rename}\NormalTok{(}\DataTypeTok{anchor_rt =}\NormalTok{ anchor, }\DataTypeTok{oneDegree_rt =}\NormalTok{ oneDegree, }
                       \DataTypeTok{premise_rt =}\NormalTok{ premise, }\DataTypeTok{twoDegree_rt =}\NormalTok{ twoDegree)}

\NormalTok{        oneRowDf =}\StringTok{ }\KeywordTok{cbind}\NormalTok{(dataSet_multiple_t_head, dataSet_multiple_t_less)}
        
        \KeywordTok{return}\NormalTok{(oneRowDf)}
\NormalTok{\}}
\end{Highlighting}
\end{Shaded}

\section{Further Data Processing}\label{further-data-processing}

Appliying the function. First getting number of columns for each data
set, so I know how many participants we have - total / 5.

Also getting name of the columns, so I know which tokens used = subject
number.

\begin{Shaded}
\begin{Highlighting}[]
\NormalTok{noCol_ts_W =}\StringTok{ }\KeywordTok{ncol}\NormalTok{(ts_wake_R)}
\NormalTok{noCol_ts_S =}\StringTok{ }\KeywordTok{ncol}\NormalTok{(ts_sleep_R)}

\NormalTok{noCol_ts_W    ## 20 columns / 5 = number of participants}
\end{Highlighting}
\end{Shaded}

\begin{verbatim}
## [1] 20
\end{verbatim}

\begin{Shaded}
\begin{Highlighting}[]
\NormalTok{noCol_ts_S    ## 10 columns / 5 = number of participants}
\end{Highlighting}
\end{Shaded}

\begin{verbatim}
## [1] 10
\end{verbatim}

\begin{Shaded}
\begin{Highlighting}[]
\NormalTok{namesCol_ts_W =}\StringTok{ }\KeywordTok{colnames}\NormalTok{(ts_wake_R)}
\NormalTok{namesCol_ts_S =}\StringTok{ }\KeywordTok{colnames}\NormalTok{(ts_sleep_R)}

\NormalTok{namesCol_ts_W}
\end{Highlighting}
\end{Shaded}

\begin{verbatim}
##  [1] "pairType_W_6"            "letterPos1_W_6"         
##  [3] "letterPos2_W_6"          "key_resp_test.corr_W_6" 
##  [5] "key_resp_test.rt_W_6"    "pairType_W_9"           
##  [7] "letterPos1_W_9"          "letterPos2_W_9"         
##  [9] "key_resp_test.corr_W_9"  "key_resp_test.rt_W_9"   
## [11] "pairType_W_11"           "letterPos1_W_11"        
## [13] "letterPos2_W_11"         "key_resp_test.corr_W_11"
## [15] "key_resp_test.rt_W_11"   "pairType_W_15"          
## [17] "letterPos1_W_15"         "letterPos2_W_15"        
## [19] "key_resp_test.corr_W_15" "key_resp_test.rt_W_15"
\end{verbatim}

\begin{Shaded}
\begin{Highlighting}[]
\CommentTok{#  [1] "pairType_W_6"            "letterPos1_W_6"          "letterPos2_W_6"         }
\CommentTok{#  [4] "key_resp_test.corr_W_6"  "key_resp_test.rt_W_6"    "pairType_W_9"           }
\CommentTok{#  [7] "letterPos1_W_9"          "letterPos2_W_9"          "key_resp_test.corr_W_9" }
\CommentTok{# [10] "key_resp_test.rt_W_9"    "pairType_W_11"           "letterPos1_W_11"        }
\CommentTok{# [13] "letterPos2_W_11"         "key_resp_test.corr_W_11" "key_resp_test.rt_W_11"  }
\CommentTok{# [16] "pairType_W_15"           "letterPos1_W_15"         "letterPos2_W_15"        }
\CommentTok{# [19] "key_resp_test.corr_W_15" "key_resp_test.rt_W_15"}
\CommentTok{# }

\CommentTok{# tokens_W = 6, 9, 11, 15}

\NormalTok{namesCol_ts_S}
\end{Highlighting}
\end{Shaded}

\begin{verbatim}
##  [1] "pairType_S_7"           "letterPos1_S_7"         "letterPos2_S_7"        
##  [4] "key_resp_test.corr_S_7" "key_resp_test.rt_S_7"   "pairType_S_8"          
##  [7] "letterPos1_S_8"         "letterPos2_S_8"         "key_resp_test.corr_S_8"
## [10] "key_resp_test.rt_S_8"
\end{verbatim}

\begin{Shaded}
\begin{Highlighting}[]
\CommentTok{#  [1] "pairType_S_7"           "letterPos1_S_7"         "letterPos2_S_7"        }
\CommentTok{#  [4] "key_resp_test.corr_S_7" "key_resp_test.rt_S_7"   "pairType_S_8"          }
\CommentTok{#  [7] "letterPos1_S_8"         "letterPos2_S_8"         "key_resp_test.corr_S_8"}
\CommentTok{# [10] "key_resp_test.rt_S_8"  }

\CommentTok{# tokens_S = 7, 8}
\end{Highlighting}
\end{Shaded}

Creating a df out of the function. Gives me total of corr responses per
pair type and average rt per pair type. Then make a proportion out of it
and combine those df, into one bigger df. Consider to write another
function.

\begin{Shaded}
\begin{Highlighting}[]
\CommentTok{# ## wake testing}
\CommentTok{# subj_6 = summary_subj_ts_f(ts_wake_R, 1, 5)}
\CommentTok{# subj_9 = summary_subj_ts_f(ts_wake_R, 6, 10)}
\CommentTok{# subj_11 = summary_subj_ts_f(ts_wake_R, 11, 15)}
\CommentTok{# subj_15 = summary_subj_ts_f(ts_wake_R, 16, 20)}
\CommentTok{# }
\CommentTok{# ## sleep testing}
\CommentTok{# subj_7 = summary_subj_ts_f(ts_sleep_R, 1, 5)}
\CommentTok{# subj_8 = summary_subj_ts_f(ts_sleep_R, 6, 10)}
\end{Highlighting}
\end{Shaded}

Now create a tibble out of those single outputs. First transpose rows
and columns. \textless{}- that's already in the function Then add
together. \textless{}- also in the function

\begin{Shaded}
\begin{Highlighting}[]
\CommentTok{# df_ts = rbind(subj_6, subj_9, subj_11, subj_15, subj_7, subj_8)}
\end{Highlighting}
\end{Shaded}

\begin{center}\rule{0.5\linewidth}{0.5pt}\end{center}

\end{document}
